%%=============================================================================
%% Inleiding
%%=============================================================================

\chapter{\IfLanguageName{dutch}{Inleiding}{Introduction}}%
\label{ch:inleiding}

In een tijdperk waar digitalisering centraal staat is het cruciaal om alles veilig te laten gebeuren, zoals digitale transacties. 
Dit onder andere bij elektronische facturen, wat van economisch belang is.
Een internationale standaard hiervoor is Peppol. Dit netwerk integreert business processen door de manier waarop informatie 
gestructureerd en uitgewisseld wordt te standaardiseren \autocite{Peppol2022}.

Scrada fungeert als een Certified Peppol Access Point. Om te integriteit van het netwerk te behouden 
en fraude te voorkomen is het essentieel voor Scrada om veilig en correct een Know Your Customer (KYC of client onboarding) uit te voeren. 
Dit houdt in dat de identiteit van de klant strict gecontroleerd moet worden binnen een business relation en moet voldoen aan belangrijke regels. 
In dit onderzoek wordt gefocust hoe dit verificatieproces voor Scrada
geautomatiseerd en geoptimaliseerd kan worden, zodat deze volledig voldoen aan de conformiteit van Peppol.


%\begin{itemize}
%  \item context, achtergrond
%  \item afbakenen van het onderwerp
%  \item verantwoording van het onderwerp, methodologie
%  \item probleemstelling
%  \item onderzoeksdoelstelling
%  \item onderzoeksvraag
%  \item \ldots
%\end{itemize}

\section{\IfLanguageName{dutch}{Probleemstelling}{Problem Statement}}%
\label{sec:probleemstelling}

Hoewel Scrada als Certified Peppol Access Point een cruciale rol speelt in veilige e-facturatie, zijn er een aantal factoren die een gevaar kunnen vormen voor de veiligheid van het Peppol-netwerk van het bedrijf.
Momenteel wordt de KYC idnentificatie manueel uitgevoerd aan de hand van een kopie van de identiteitskaart van de klant die gecontroleerd wordt op de openbare KBO-databank \ (https://kbopub.economie.fgov.be/kbopub
/zoeknaamfonetischform.html). Dit proces is problematisch voor de primaire doelgroep, de IT- en compliance-afdeling van Scrada. Dit omwille van verschillende redenen:
\begin{itemize}
  \item \textbf{GDPR-risico's:} Het verwerken van volledige identiteitskaarten is tegenstrijdig met het principe van dataminimalisatie. 
  Daarnaast dwingt de huidige werkwijze Scrada om meer persoonsgegevens te verwerken dan noodzakelijk is, wat de naleving van de GDPR-wetgeving bemoeilijkt en onnodige privacyrisico's met zich meebrengt.
  \item \textbf{Efficiëntie:} De manuele handelingen maken het proces traag en foutgevoelig bij een stijgend aantal klanten. Daarnaast vermoeilijkt dit ook de schaalbaarheid van het bedrijfsproces.
  \item \textbf{Fraudegevoeligheid:} Handmatige verificatie is minder bestand tegen geavanceerde vormen van identiteitsfraude dan geautomatiseerde, cryptografische methoden.
\end{itemize}

\section{\IfLanguageName{dutch}{Onderzoeksvraag}{Research question}}%
\label{sec:onderzoeksvraag}

De centrale onderzoeksvraag van deze bachelorproef luidt als volgt:
\begin{quote}
    \textit{"Hoe kan het KYC-proces voor identiteitsverificatie op het Peppol-platform van Scrada geautomatiseerd worden op een GDPR-conforme, fraudeveilige en kostenefficiënte manier?"}
\end{quote}

De deelvragen die bijdragen aan het vormen van een antwoord op deze centrale onderzoeksvraag zijn:
\subsection{Deelvragen probleemdomein} 
\begin{itemize}
    \item {{Hoe verloopt het huidige KYC-proces bij Scrada op het Peppol-platform?}}
    \item {{Welke problemen ervaart Scrada bij het huidige KYC-proces?}}
    \item {{Op welke stappen in het proces kan automatisering toegepast worden?}}
    \item {{Welke technologieën kunnen gebruikt worden om het KYC-proces bij Scrada te automatiseren?}}
    \item {{Welke gegevens van bedrijven worden verzameld tijdens het KYC-proces en hoe gaat Scrada hiermee om?}}
    \item {{Wat zijn de kenmerken van een GDPR- \- conforme authenticatieprocedure en welke voordelen heeft deze tegenover de huidige methode bij Scrada?}}
    \item {{Welke frauderisico’s bestaan er bij het huidige KYC-proces?}}
    \item {{Welke kosten draagt het huidige KYC-proces bij Scrada?}}
\end{itemize}

\subsection{Deelvragen oplossingsdomein} 
\begin{itemize}
    \item {{Hoe kunnen de geautomatiseerde KYC- \- processen conform blijven aan de GDPR-regels?}}
    \item {{Welke risico’s bestaan bij het verwerken van persoonsgegevens in een geautomatiseerd KYC-proces?}}
    \item {{Welke maatregelen/technologieën kunnen worden ingezet om fraude bij het geautomatiseerde KYC-proces te voorkomen?}}
    \item {{Hoe kan automatisering bijdragen aan kostenreductie zonder nadelig te zijn voor GDPR en fraudeveiligheid?}}
\end{itemize}

\section{\IfLanguageName{dutch}{Onderzoeksdoelstelling}{Research objective}}%
\label{sec:onderzoeksdoelstelling}

De doelstelling van dit onderzoek is om een geautomatiseerd KYC-proces te ontwerpen en te implementeren dat voldoet aan de GDPR-regelgeving, bestand is tegen fraude en kostenefficiënt is voor Scrada. 
Dit omvat het identificeren van geschikte technologieën, het ontwikkelen van een prototype en het evalueren van de effectiviteit van het nieuwe proces in vergelijking met de huidige methoden.
Bijgevolg is het resultaat een Proof-of-Concept (PoC) die aantoont dat identiteitsverificatie sneller en veiliger kan zonder de huidige privacyrisico's.

De criteria voor een succesvolle Proof-of-Concept zijn:
\begin{itemize}
    \item \textbf{Efficiëntie:} Een reductie van de gemiddelde KYC-doorlooptijd met minimaal 50\%. Daarnaast moet de oplossing ook kostenefficiënt zijn, met een duidelijke kosten-batenanalyse die aantoont dat de implementatie van het geautomatiseerde KYC-proces financieel haalbaar is voor Scrada.
    \item \textbf{Compliance:} Het elimineren van de noodzaak om volledige kopieën van identiteitskaarten op te slaan, conform aan de GDPR-principes.
    \item \textbf{Validatie:} Een werkend prototype dat succesvol communiceert met externe databronnen (zoals de KBO API) en de integriteit van de gebruiker verifieert.
\end{itemize}


\section{\IfLanguageName{dutch}{Opzet van deze bachelorproef}{Structure of this bachelor thesis}}%
\label{sec:opzet-bachelorproef}

% Het is gebruikelijk aan het einde van de inleiding een overzicht te
% geven van de opbouw van de rest van de tekst. Deze sectie bevat al een aanzet
% die je kan aanvullen/aanpassen in functie van je eigen tekst.

De rest van deze bachelorproef is als volgt opgebouwd:

TODO de opbouw laten kloppen
In Hoofdstuk~\ref{ch:stand-van-zaken} wordt de theoretische basis gelegd. Hierbij wordt de huidige wetgeving rond de GDPR, de werking van het Peppol-netwerk en moderne KYC-technologieën onderzocht. Dit vormt het referentiekader waarbinnen de automatisering moet plaatsvinden.
In Hoofdstuk~\ref{ch:analyse} vindt de analyse van de huidige werking bij Scrada plaats. Hier worden de bestaande manuele handelingen in kaart gebracht en getoetst aan de GDPR-principes om te kunnen identificeren waar de grootste bottlenecks liggen en waar automatisering efficiënt kan toegepast worden.
In Hoofdstuk~\ref{ch:methodologie} wordt de methodologie toegelicht en worden de gebruikte onderzoekstechnieken besproken om een antwoord te kunnen formuleren op de onderzoeksvragen.
In Hoofdstuk~\ref{ch:implementatie} wordt op basis van die analyse het ontwerp en de ontwikkeling van de Proof-of-Concept beschreven, waarbij de technische keuzes voor automatisering worden uitgelicht.
Hoofdstuk~\ref{ch:evaluatie} bespreekt de resultaten van het onderzoek. Hierbij wordt de geautomatiseerde flow vergeleken met de huidige werkwijze bij Scrada om aan te tonen dat het proces sneller, veiliger en privacyvriendelijker is geworden.
In Hoofdstuk~\ref{ch:conclusie} tenslotte, wordt de conclusie gegeven en een antwoord geformuleerd op de onderzoeksvragen. Daarbij wordt ook een aanzet gegeven voor toekomstig onderzoek binnen dit domein.