%%=============================================================================
%% Inleiding
%%=============================================================================

\chapter{\IfLanguageName{dutch}{Inleiding}{Introduction}}%
\label{ch:inleiding}

In een tijdperk waar digitalisering centraal staat is het cruciaal om alles veilig te laten gebeuren, zoals digitale transacties. 
Dit onder andere bij elektronische facturen, wat van economisch belang is.
Een internationale standaard hiervoor is Peppol. Dit netwerk integreert business processen door de manier waarop informatie 
gestructureerd en uitgewisseld wordt te standaardiseren \autocite{Peppol2022}.

Scrada fungeert als een Certified Peppol Access Point. Om te integriteit van het netwerk te behouden 
en fraude te voorkomen is het essentieel voor Scrada om veilig en correct een Know Your Customer (KYC of client onboarding) uit te voeren. 
Dit houdt in dat de identiteit van de klant strict gecontroleerd moet worden binnen een business relation en moet voldoen aan belangrijke regels. 
In dit onderzoek wordt gefocust hoe dit verificatieproces voor Scrada
geautomatiseerd en geoptimaliseerd kan worden, zodat deze volledig voldoen aan de conformiteit van Peppol.


\begin{itemize}
  \item context, achtergrond
  \item afbakenen van het onderwerp
  \item verantwoording van het onderwerp, methodologie
  \item probleemstelling
  \item onderzoeksdoelstelling
  \item onderzoeksvraag
  \item \ldots
\end{itemize}

\section{\IfLanguageName{dutch}{Probleemstelling}{Problem Statement}}%
\label{sec:probleemstelling}

Hoewel Scrada als Certified Peppol Access Point een cruciale rol speelt in veilige e-facturatie, zijn er een aantal factoren die een gevaar kunnen vormen voor de veiligheid van het Peppol-netwerk van het bedrijf.
Momenteel wordt de KYC idnentificatie manueel uitgevoerd aan de hand van een kopie van de identiteitskaart van de klant die gecontroleerd wordt op de openbare KBO-databank (https://kbopub.economie.fgov.be/kbopub
/zoeknaamfonetischform.html). Dit proces is problematisch voor de primaire doelgroep, de IT- en compliance-afdeling van Scrada. Dit omwille van verschillende redenen:
\begin{itemize}
  \item \textbf{GDPR-risico's:} Het verwerken van volledige identiteitskaarten is tegenstrijdig met het principe van dataminimalisatie. 
  Daarnaast dwingt de huidige werkwijze Scrada om meer persoonsgegevens te verwerken dan noodzakelijk is, wat de naleving van de GDPR-wetgeving bemoeilijkt en onnodige privacyrisico's met zich meebrengt.
  \item \textbf{Efficiëntie:} De manuele handelingen maken het proces traag en foutgevoelig bij een stijgend aantal klanten. Daarnaast vermoeilijkt dit ook de schaalbaarheid van het bedrijfsproces.
  \item \textbf{Fraudegevoeligheid:} Handmatige verificatie is minder bestand tegen geavanceerde vormen van identiteitsfraude dan geautomatiseerde, cryptografische methoden.
\end{itemize}

\section{\IfLanguageName{dutch}{Onderzoeksvraag}{Research question}}%
\label{sec:onderzoeksvraag}

Wees zo concreet mogelijk bij het formuleren van je onderzoeksvraag. Een onderzoeksvraag is trouwens iets waar nog niemand op dit moment een antwoord heeft (voor zover je kan nagaan). Het opzoeken van bestaande informatie (bv. ``welke tools bestaan er voor deze toepassing?'') is dus geen onderzoeksvraag. Je kan de onderzoeksvraag verder specifiëren in deelvragen. Bv.~als je onderzoek gaat over performantiemetingen, dan 

\section{\IfLanguageName{dutch}{Onderzoeksdoelstelling}{Research objective}}%
\label{sec:onderzoeksdoelstelling}

Wat is het beoogde resultaat van je bachelorproef? Wat zijn de criteria voor succes? Beschrijf die zo concreet mogelijk. Gaat het bv.\ om een proof-of-concept, een prototype, een verslag met aanbevelingen, een vergelijkende studie, enz.

\section{\IfLanguageName{dutch}{Opzet van deze bachelorproef}{Structure of this bachelor thesis}}%
\label{sec:opzet-bachelorproef}

% Het is gebruikelijk aan het einde van de inleiding een overzicht te
% geven van de opbouw van de rest van de tekst. Deze sectie bevat al een aanzet
% die je kan aanvullen/aanpassen in functie van je eigen tekst.

De rest van deze bachelorproef is als volgt opgebouwd:

In Hoofdstuk~\ref{ch:stand-van-zaken} wordt een overzicht gegeven van de stand van zaken binnen het onderzoeksdomein, op basis van een literatuurstudie.

In Hoofdstuk~\ref{ch:methodologie} wordt de methodologie toegelicht en worden de gebruikte onderzoekstechnieken besproken om een antwoord te kunnen formuleren op de onderzoeksvragen.

% TODO: Vul hier aan voor je eigen hoofstukken, één of twee zinnen per hoofdstuk

In Hoofdstuk~\ref{ch:conclusie}, tenslotte, wordt de conclusie gegeven en een antwoord geformuleerd op de onderzoeksvragen. Daarbij wordt ook een aanzet gegeven voor toekomstig onderzoek binnen dit domein.