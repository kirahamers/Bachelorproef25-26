\chapter{\IfLanguageName{dutch}{Stand van zaken}{State of the art}}%
\label{ch:stand-van-zaken}

% Tip: Begin elk hoofdstuk met een paragraaf inleiding die beschrijft hoe
% dit hoofdstuk past binnen het geheel van de bachelorproef. Geef in het
% bijzonder aan wat de link is met het vorige en volgende hoofdstuk.

% Pas na deze inleidende paragraaf komt de eerste sectiehoofding.

In het vorige hoofdstuk is de noodzaak voor een efficiënter en veiliger KYC-proces bij Scrada aangetoond. 
Om een correct werkende automatisering te kunnen ontwerpen, is een grondig begrip van zowel de juridische voorwaarden als de beschikbare technologische bouwstenen essentieel. 
Dit hoofdstuk onderzoekt het Peppol-ecosysteem, de strenge vereisten van de GDPR en de moderne technologieën die identiteitsverificatie kunnen digitaliseren en efficiënter maken. 
Dit vormt de theoretische fundering voor de Proof-of-Concept in de latere hoofdstukken.

\label{sec:Juridisch Kader: Het spanningsveld tussen KYC en GDPR}
\label{sec:juridisch-kader}
Bij KYC-processen worden per definitie persoonsgegevens verwerkt. 
Het is een kritieke procedure die financiële instellingen en bedrijven helpt om de identiteit van hun klanten te verifiëren, 
wat essentieel is voor het voorkomen en identificeren van fraude, witwassen van geld en andere illegale activiteiten.
Hierdoor is het KYC-proces internationaal gestandaardiseerd.

Echter, de identificatie en privacyrechten van de klant moeten in evenwicht worden gebracht, wat een uitdaging vormt voor bedrijven zoals Scrada die aan de GDPR-regelgeving moeten voldoen.
In deze sectie worden de belangrijkste juridische principes van de GDPR besproken die relevant zijn voor het KYC-proces, 
evenals de uitdagingen en mogelijke oplossingen voor het implementeren van een geautomatiseerd KYC-systeem dat voldoet aan deze regelgeving.

\subsection{De Belgische Anti-witwaswetgeving (AML)}
Bedrijven in België zijn wettelijk verplicht om een KYC-proces te implementeren dat voldoet aan de Belgische 
anti-witwaswetgeving (AML) \autocite{FODFinancien2026}.
Deze wetgeving vereist dat bedrijven de identiteit van hun klanten verifiëren, 
de herkomst van fondsen controleren en verdachte activiteiten melden aan de autoriteiten. 
De AML-regelgeving is ontworpen om het witwassen van geld, de financiering van terrorisme (de ‘wet AML’) en het verspreiden van massavernietigingswapens
te voorkomen en te bestrijden, en legt strenge regels op aan bedrijven om ervoor te zorgen dat ze niet onbedoeld betrokken raken bij illegale activiteiten. 
Het niet naleven van deze wetgeving kan leiden tot zware boetes en reputatieschade, wat de noodzaak benadrukt voor bedrijven zoals Scrada om een veilig en compliant KYC-proces te implementeren.
Daarom moet hierop de nadruk gezet worden bij het ontwerpen van een geautomatiseerd KYC-systeem, waarbij zowel de juridische vereisten als de operationaliteit van belang zijn.
Hierbij speelt de Administratie van de Thesaurie een belangrijke rol (een onderdeel van de FOD Financiën) die 
binnen de federale structuur verantwoordelijk is voor onder meer het beheer van het UBO-register, wat essentieel is voor het identificeren van de uiteindelijke begunstigden 
van een vennootschap en de opvolging van financiële toepassingen (naast het welgekende beheer van de Belgische schatkist) \autocite{VlaamseOverheid2026}.

\subsection{Principes van de GDPR}
Om een geautomatiseerd KYC-proces te creëren dat juridisch conform is aan de GDPR, 
moet er specifiek rekening worden gehouden met de kernprincipes van de Europese privacywetgeving \autocite{EuropeanUnion2016}. 
Voor dit onderzoek staan de volgende artikelen centraal:
\autocite{EuropeanUnion2016}:
\begin{itemize}
    \item \textbf{Rechtmatigheid, eerlijkheid en transparantie (Art. 5.1a):} Alle gegevensverwerking moet gebaseerd zijn op een geldige juridische grondslag, eerlijk worden uitgevoerd en transparant zijn voor de betrokken personen.
    \item \textbf{Doelbinding (Art. 5.1b):} Persoonsgegevens mogen alleen worden verzameld voor specifieke, expliciete en legitieme doeleinden en mogen niet verwerkt worden voor doeleinden die los hiervan staan.
    \item \textbf{Dataminimalisatie (Art. 5.1c):} Dit principe vereist dat alleen de persoonsgegevens die noodzakelijk zijn voor het specifieke doel van het KYC-proces worden verzameld en verwerkt.
    \item \textbf{Opslagbeperking (Art. 5.1e):} Persoonsgegevens mogen niet langer worden bewaard dan noodzakelijk is voor de doeleinden waarvoor ze zijn verzameld.
    \item \textbf{Rechten van betrokkenen (Art. 12-23):} Klanten hebben verschillende rechten onder de GDPR, zoals het recht op toegang tot hun gegevens, het recht op rectificatie, het recht op gegevenswissing en het recht om bezwaar te maken tegen bepaalde vormen van verwerking. Het KYC-systeem moet deze rechten respecteren en faciliteren.
    \item \textbf{Geautomatiseerde besluitvorming (Art. 22):} Indien het KYC-proces geautomatiseerde besluitvorming omvat, zoals het automatisch goedkeuren of weigeren van een klant, zijn er extra waarborgen nodig om de rechten en vrijheden van de betrokken personen te beschermen.
    \item \textbf{Privacy by Design (Art. 25):} Bedrijven moeten vanaf het begin van het ontwerp van hun KYC-systeem rekening houden beveiliging en privacy te garanderen, bijvoorbeeld door gegevens te pseudonimiseren of te versleutelen en door strikte toegangscontroles in te voeren.
\end{itemize}

Een van de punten die zorgt voor het spanningsveld tussen KYC en GDPR is de bewaartijd van gegevens (Opslagbeperking (Art. 5.1e)).
Hoewel de AML-regelgeving vereist dat bepaalde gegevens gedurende een specifieke periode worden bewaard,
moet dit in overeenstemming zijn met het GDPR-principe van opslagbeperking.
Bedrijven moeten daarom een duidelijk beleid hebben voor het bewaren en verwijderen van KYC-gegevens, om zowel aan de AML-vereisten als aan de GDPR te voldoen.

\label{sec:Het gestandaardiseerde KYC-proces}
\label{sec:kyc-proces}
Volgens de \textcite{LSEG2024} bestaat een effectief proces uit drie essentiële stappen: 

\begin{enumerate}
    \item \textbf{Customer Identification Program (CIP):} In de eerste fase van het KYC-proces vindt de onboarding plaats. Hier wordt de essentiële informatie van de klant verzameld met behulp van betrouwbare bronnen zoals paspoorten, rijbewijzen of bedrijfsregistraties.
    \item \textbf{Customer Due Diligence (CDD):} In deze fase wordt de verzamelde informatie geverifieerd aan de hand van betrouwbare, onafhankelijke bronnen. Er wordt onderzocht of het bedrijf legitiem is en of er eventuele risico's verbonden zijn aan de zakelijke relatie, bijvoorbeeld door te controleren op sanctielijsten en te bekijken wat de intenties van de klant zijn.
    Hierbij wordt gebruik gemaakt van de LSEG World-Check-database, die uitgebreide informatie bevat over individuen en bedrijven die mogelijk sancties of andere risico's vormen. Er kan ook gebruik worden gemaakt van
andere bronnen zoals de Kruispuntbank van Ondernemingen (KBO), waar alle Belgische ondernemingen en hun identificatiegegevens te vinden zijn en het UBO-register (Ultimate Beneficial Owner), dat inzicht geeft in de uiteindelijke belanghebbenden van een bedrijf.
    \item \textbf{Ongoing Monitoring:} KYC stopt niet na de onboarding maar wordt tijdens de hele customer journey voortgezet. De zakelijke relatie moet periodiek gecontroleerd worden om te garanderen dat de risicoprofielen actueel blijven en verdachte transactiepatronen tijdig gedetecteerd worden.
\end{enumerate}

Het manueel uitvoeren van deze stappen is tijdrovend, foutgevoelig en moeilijk schaalbaar, vooral voor bedrijven die een groot aantal klanten moeten onboarden.
De stappen die binnen een manueel KYC-proces worden ondernomen, verlopen als volgt \autocite{SDKFinance2024}:
\begin{itemize}
    \item De klant stuurt de nodige documenten en de status wordt omgezet naar "pending" in het systeem (ervoor was dit "none"). 
    \item Een KYC-medewerken keurt de documenten goed of af. Dit kan alleen gedaan worden als alle documenten aanwezig zijn.
    \item Wanneer de documenten goedgekeurd zijn, wordt de status van de klant omgezet naar "approved". Indien er documenten ontbreken of niet overeenkomen, wordt de status teruggezet naar "none" en wordt de klant geïnformeerd dat hij opnieuw moet beginnen.

\subsection{Huidige werkwijze bij Scrada}
De huidige procedure binnen Scrada is manueel proces. Hoewel de gegevensverzameling digitaal verloopt, is de verificatiestap volledig afhankelijk van handmatige acties. Het proces ziet er als volgt uit:

\begin{enumerate}
    \item \textbf{Data invoer:} De cliënt voert bedrijfsgegevens in en uploadt een kopie van de identiteitskaart, waarbij een notificatie wordt gestuurd naar Scrada.
    \item \textbf{Handmatige verificatie:} Een medewerker voert manueel een opzoeking uit in de Kruispuntbank van Ondernemingen (KBO) en vergelijkt de statutaire bestuurders met de geüploade identiteitsbewijzen.
    \item \textbf{Statusbeheer:} Bij een correcte match wordt de entiteit handmatig gevalideerd in het systeem. Indien de namen niet overeenkomen, wordt alles gereset en wordt het bedrijf op de hoogte gebracht om de gegevens opnieuw in te voeren.
    \item \textbf{Dataretentie:} Na enige tijd wordt de kopie van de identiteitskaart handmatig of na een ingestelde periode gewist om te voldoen aan de GDPR-richtlijnen.
\end{enumerate}

TODO: geautomiseerd proces beschrijven

In plaats van PDF-documenten te verwerken, communiceert een systeem via API's. Een voorbeeld van gestructureerde JSON-data die na een dergelijke controle wordt ontvangen, 
is te zien in Listing \ref{lst:api-response}.

\begin{listing}[ht]
\begin{minted}{json}
{
  "enterprise_number": "0123.456.789",
  "status": "Active",
  "sanction_check": "Cleared",
  "ubo_identified": true,
  "risk_score": "Low"
}
\end{minted}
\caption{Voorbeeld van een JSON API-response voor KYC-verificatie.}
\label{lst:api-response}
\end{listing}

\begin{figure}[ht]
  \centering
  \includegraphics[width=0.8\textwidth]{peppol_four_corner_model.png}
  \caption{Het Peppol Four-Corner Model waarbij Corner 2 verantwoordelijk is voor de KYC-verificatie.}
  \label{fig:peppol-model}
\end{figure}

In dit model fungeert Scrada als Corner 2. Omdat Peppol een 'Trust Community' is, rust de verantwoordelijkheid bij het Access Point om te garanderen dat de verzender legitiem is, om zo factuurfraude te voorkomen.

\begin{table}[ht]
  \centering
  \begin{tabular}{@{}lll@{}}
    \toprule
    \textbf{Kenmerk} & \textbf{Handmatig Proces} & \textbf{Geautomatiseerd (Scrada)} \\ \midrule
    Identificatie    & Kopie ID via e-mail       & Digitale ID (itsme/eID)           \\
    Bedrijfscheck    & Zoeken in KBO-website     & Real-time API-koppeling           \\
    Sanctiecontrole  & Handmatige Google-check   & LSEG World-Check API              \\
    Foutenmarge      & Hoog (menselijk)          & Minimaal (data-gedreven)          \\
    Bewaarplicht     & Papieren/PDF archief      & Versleutelde Cloud-opslag         \\ \bottomrule
  \end{tabular}
  \caption{Vergelijking tussen handmatige KYC en de geautomatiseerde aanpak.}
  \label{tab:kyc-vergelijking}
\end{table}