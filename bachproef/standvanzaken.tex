\usetikzlibrary{arrows.meta, shapes.geometric, calc}

\tikzset{
    start/.style={circle, draw, thick, fill=green!20, minimum size=0.7cm},
    end/.style={circle, draw, ultra thick, fill=red!20, minimum size=0.7cm},
    task/.style={rectangle, draw, thick, fill=blue!5, text width=2.3cm, align=center, minimum height=1cm, rounded corners=2pt, font=\footnotesize},
    gateway/.style={diamond, draw, thick, fill=yellow!20, text width=1.2cm, align=center, inner sep=0pt, font=\footnotesize},
    arrow/.style={thick, ->, >=stealth}
}

\chapter{\IfLanguageName{dutch}{Stand van zaken}{State of the art}}%
\label{ch:stand-van-zaken}

% Tip: Begin elk hoofdstuk met een paragraaf inleiding die beschrijft hoe
% dit hoofdstuk past binnen het geheel van de bachelorproef. Geef in het
% bijzonder aan wat de link is met het vorige en volgende hoofdstuk.

% Pas na deze inleidende paragraaf komt de eerste sectiehoofding.

In het vorige hoofdstuk is de noodzaak voor een efficiënter en veiliger KYC-proces bij Scrada aangetoond. 
Om een correct werkende automatisering te kunnen ontwerpen, is een grondig begrip van zowel de juridische voorwaarden als de beschikbare technologische bouwstenen essentieel. 
Dit hoofdstuk onderzoekt het Peppol-ecosysteem, de strenge vereisten van de GDPR en de moderne technologieën die identiteitsverificatie kunnen digitaliseren en efficiënter maken. 
Dit vormt de theoretische fundering voor de Proof-of-Concept in de latere hoofdstukken.

\label{sec:het Peppol-ecosysteem en de rol van Scrada}
Voordat de specifieke verificatieprocessen worden geanalyseerd, 
is het noodzakelijk om de omgeving waarin Scrada opereert te kaderen. 
Scrada is een gecertificeerd Peppol Access Point, binnen de sector ook wel een Service Provider genoemd. 
Peppol is hierbij een 'enabler' die de integratie van bedrijfsprocessen tussen organisaties en overheden wereldwijd 
vergemakkelijkt door middel van open standaarden \autocite{OpenPeppolMain2024}.

\subsection{Het Four-Corner Model}
De architectuur van Peppol is gebaseerd op het \textit{Four-Corner Model} (zie Figuur \ref{fig:peppol-model}). 
Dit model is ontworpen om de beperkingen van traditionele, gesloten 'three-corner' netwerken te doorbreken, 
waarbij zenders en ontvangers verplicht bij dezelfde provider aangesloten moesten zijn \autocite{PeppolEndUsers2024}. 
In de Peppol-architectuur zijn de volgende vier partijen betrokken:
\begin{itemize}
    \item \textbf{Corner 1 (De Verzender):} Het bedrijf of de organisatie die de elektronische factuur verstuurt (de cliënt van Scrada).
    \item \textbf{Corner 2 (Access Point Verzender):} De \textit{Service Provider} van de verzender (\textbf{Scrada}), verantwoordelijk voor het veilig op het netwerk brengen van de documenten.
    \item \textbf{Corner 3 (Access Point Ontvanger):} De \textit{Service Provider} van de ontvanger, die de documenten valideert en aflevert.
    \item \textbf{Corner 4 (De Ontvanger):} De uiteindelijke ontvanger van de factuur, zoals een overheidsinstantie of een andere onderneming.
\end{itemize}

\subsection{Scrada als Poortwachter (Corner 2)}
Binnen dit decentrale model draagt Corner 2 een aanzienlijke verantwoordelijkheid. 
Omdat het Peppol-netwerk fundamenteel gebaseerd is op gedeeld onderling vertrouwen (trust), 
gaan Corner 3 en 4 er impliciet van uit dat Corner 2 de identiteit van Corner 1 sluitend heeft geverifieerd 
voordat toegang tot het netwerk wordt verleend. 
"Omdat Scrada een officieel erkende Service Provider is, moet het bedrijf zich strikt houden aan het Peppol Interoperability Framework (de werkelijke naam van Peppol). 
Die zorgt ervoor dat elektronische documenten gestandaardiseerd worden voor validatie en veilige uitwisseling 
tussen Service Providers wereldwijd. Hierbij wordt gebruikgemaakt van de internationale ISO/IEC 19845-standaard (Universal Business Language), waardoor facturen over grenzen en sectoren heen leesbaar blijven zonder dat zender en ontvanger dezelfde software hoeven te gebruiken \autocite{PeppolFramework2024}.
Dit is een verplichte set van juridische en technische afspraken die ervoor zorgt dat alle partijen in het netwerk op dezelfde, 
veilige manier samenwerken \autocite{PeppolServiceProviders2024}."
Indien een Access Point tekortschiet hun KYC verplichtingen niet vervult, ontstaat het risico dat kwaadwillige actoren toegang krijgen tot het netwerk om op grote schaal frauduleuze facturen te verspreiden. 
De KYC-procedure is volgens de standaarden van OpenPeppol dan ook geen administratieve formaliteit, maar een belangrijke beveiligingslaag. 
Deze laag garandeert de authenticiteit van de verzender en beschermt de integriteit van het volledige wereldwijde ecosysteem tegen financiële criminaliteit en identiteitsfraude \autocite{PeppolEndUsers2024}.

\begin{figure}[ht]
  \centering
  \includegraphics[width=0.8\textwidth]{peppol_four_corner_model.png}
  \caption{Het Peppol Four-Corner Model waarbij Corner 2 verantwoordelijk is voor de KYC-verificatie.}
  \label{fig:peppol-model}
\end{figure}

\label{sec:Juridisch Kader: Het spanningsveld tussen KYC en GDPR}
\label{sec:juridisch-kader}
Bij KYC-processen worden per definitie persoonsgegevens verwerkt. 
Het is een kritieke procedure die financiële instellingen en bedrijven helpt om de identiteit van hun klanten te verifiëren, 
wat essentieel is voor het voorkomen en identificeren van fraude, witwassen van geld en andere illegale activiteiten.
Hierdoor is het KYC-proces internationaal gestandaardiseerd.

Echter, de identificatie en privacyrechten van de klant moeten in evenwicht worden gebracht, wat een uitdaging vormt voor bedrijven zoals Scrada die aan de GDPR-regelgeving moeten voldoen.
In deze sectie worden de belangrijkste juridische principes van de GDPR besproken die relevant zijn voor het KYC-proces, 
evenals de uitdagingen en mogelijke oplossingen voor het implementeren van een geautomatiseerd KYC-systeem dat voldoet aan deze regelgeving.

\subsection{De Belgische Anti-witwaswetgeving (AML)}
Bedrijven in België zijn wettelijk verplicht om een KYC-proces te implementeren dat voldoet aan de Belgische 
anti-witwaswetgeving (AML) \autocite{FODFinancien2026}.
Deze wetgeving vereist dat bedrijven de identiteit van hun klanten verifiëren, 
de herkomst van fondsen controleren en verdachte activiteiten melden aan de autoriteiten. 
De AML-regelgeving is ontworpen om het witwassen van geld, de financiering van terrorisme (de ‘wet AML’) en het verspreiden van massavernietigingswapens
te voorkomen en te bestrijden, en legt strenge regels op aan bedrijven om ervoor te zorgen dat ze niet onbedoeld betrokken raken bij illegale activiteiten. 
Het niet naleven van deze wetgeving kan leiden tot zware boetes en reputatieschade, wat de noodzaak benadrukt voor bedrijven zoals Scrada om een veilig en compliant KYC-proces te implementeren.
Daarom moet hierop de nadruk gezet worden bij het ontwerpen van een geautomatiseerd KYC-systeem, waarbij zowel de juridische vereisten als de operationaliteit van belang zijn.
Hierbij speelt de Administratie van de Thesaurie een belangrijke rol (een onderdeel van de FOD Financiën) die 
binnen de federale structuur verantwoordelijk is voor onder meer het beheer van het UBO-register, wat essentieel is voor het identificeren van de uiteindelijke begunstigden 
van een vennootschap en de opvolging van financiële toepassingen (naast het welgekende beheer van de Belgische schatkist) \autocite{VlaamseOverheid2026}.

\subsection{Principes van de GDPR}
Om een geautomatiseerd KYC-proces te creëren dat juridisch conform is aan de GDPR, 
moet er specifiek rekening worden gehouden met de kernprincipes van de Europese privacywetgeving \autocite{EuropeanUnion2016}. 
Voor dit onderzoek staan de volgende artikelen centraal:
\autocite{EuropeanUnion2016}:
\begin{itemize}
    \item \textbf{Rechtmatigheid, eerlijkheid en transparantie (Art. 5.1a):} Alle gegevensverwerking moet gebaseerd zijn op een geldige juridische grondslag, eerlijk worden uitgevoerd en transparant zijn voor de betrokken personen.
    \item \textbf{Doelbinding (Art. 5.1b):} Persoonsgegevens mogen alleen worden verzameld voor specifieke, expliciete en legitieme doeleinden en mogen niet verwerkt worden voor doeleinden die los hiervan staan.
    \item \textbf{Dataminimalisatie (Art. 5.1c):} Dit principe vereist dat alleen de persoonsgegevens die noodzakelijk zijn voor het specifieke doel van het KYC-proces worden verzameld en verwerkt.
    \item \textbf{Opslagbeperking (Art. 5.1e):} Persoonsgegevens mogen niet langer worden bewaard dan noodzakelijk is voor de doeleinden waarvoor ze zijn verzameld.
    \item \textbf{Rechten van betrokkenen (Art. 12-23):} Klanten hebben verschillende rechten onder de GDPR, zoals het recht op toegang tot hun gegevens, het recht op rectificatie, het recht op gegevenswissing en het recht om bezwaar te maken tegen bepaalde vormen van verwerking. Het KYC-systeem moet deze rechten respecteren en faciliteren.
    \item \textbf{Geautomatiseerde besluitvorming (Art. 22):} Indien het KYC-proces geautomatiseerde besluitvorming omvat, zoals het automatisch goedkeuren of weigeren van een klant, zijn er extra waarborgen nodig om de rechten en vrijheden van de betrokken personen te beschermen.
    \item \textbf{Privacy by Design (Art. 25):} Bedrijven moeten vanaf het begin van het ontwerp van hun KYC-systeem rekening houden beveiliging en privacy te garanderen, bijvoorbeeld door gegevens te pseudonimiseren of te versleutelen en door strikte toegangscontroles in te voeren.
\end{itemize}

Een van de punten die zorgt voor het spanningsveld tussen KYC en GDPR is de bewaartijd van gegevens (Opslagbeperking (Art. 5.1e)).
Hoewel de AML-regelgeving vereist dat bepaalde gegevens gedurende een specifieke periode worden bewaard,
moet dit in overeenstemming zijn met het GDPR-principe van opslagbeperking.
Bedrijven moeten daarom een duidelijk beleid hebben voor het bewaren en verwijderen van KYC-gegevens, om zowel aan de AML-vereisten als aan de GDPR te voldoen.

\label{sec:Het gestandaardiseerde KYC-proces}
\label{sec:kyc-proces}
Volgens de \textcite{LSEG2024} bestaat een effectief proces uit drie essentiële stappen: 

\begin{enumerate}
    \item \textbf{Customer Identification Program (CIP):} In de eerste fase van het KYC-proces vindt de onboarding plaats. Hier wordt de essentiële informatie van de klant verzameld met behulp van betrouwbare bronnen zoals paspoorten, rijbewijzen of bedrijfsregistraties.
    \item \textbf{Customer Due Diligence (CDD):} In deze fase wordt de verzamelde informatie geverifieerd aan de hand van betrouwbare, onafhankelijke bronnen. Er wordt onderzocht of het bedrijf legitiem is en of er eventuele risico's verbonden zijn aan de zakelijke relatie, bijvoorbeeld door te controleren op sanctielijsten en te bekijken wat de intenties van de klant zijn.
    Hierbij wordt gebruik gemaakt van de LSEG World-Check-database, die uitgebreide informatie bevat over individuen en bedrijven die mogelijk sancties of andere risico's vormen. Er kan ook gebruik worden gemaakt van
andere bronnen zoals de Kruispuntbank van Ondernemingen (KBO), waar alle Belgische ondernemingen en hun identificatiegegevens te vinden zijn en het UBO-register (Ultimate Beneficial Owner), dat inzicht geeft in de uiteindelijke belanghebbenden van een bedrijf.
    \item \textbf{Ongoing Monitoring:} KYC stopt niet na de onboarding maar wordt tijdens de hele customer journey voortgezet. De zakelijke relatie moet periodiek gecontroleerd worden om te garanderen dat de risicoprofielen actueel blijven en verdachte transactiepatronen tijdig gedetecteerd worden.
\end{enumerate}

Het manueel uitvoeren van deze stappen is tijdrovend, foutgevoelig en moeilijk schaalbaar, vooral voor bedrijven die een groot aantal klanten moeten onboarden.
De stappen die binnen een manueel KYC-proces worden ondernomen, verlopen als volgt \autocite{SDKFinance2024}:
\begin{itemize}
    \item De klant stuurt de nodige documenten en de status wordt omgezet naar "pending" in het systeem (ervoor was dit "none"). 
    \item Een KYC-medewerken keurt de documenten goed of af. Dit kan alleen gedaan worden als alle documenten aanwezig zijn.
    \item Wanneer de documenten goedgekeurd zijn, wordt de status van de klant omgezet naar "approved". Indien er documenten ontbreken of niet overeenkomen, wordt de status teruggezet naar "none" en wordt de klant geïnformeerd dat hij opnieuw moet beginnen.

\subsection{Huidige werkwijze bij Scrada}
De huidige procedure binnen Scrada is manueel proces. Hoewel de gegevensverzameling digitaal verloopt, is de verificatiestap volledig afhankelijk van handmatige acties. Het proces ziet er als volgt uit:

\begin{enumerate}
    \item \textbf{Data invoer:} De cliënt voert bedrijfsgegevens in en uploadt een kopie van de identiteitskaart, waarbij een notificatie wordt gestuurd naar Scrada.
    \item \textbf{Handmatige verificatie:} Een medewerker voert manueel een opzoeking uit in de Kruispuntbank van Ondernemingen (KBO) en vergelijkt de statutaire bestuurders met de geüploade identiteitsbewijzen.
    \item \textbf{Statusbeheer:} Bij een correcte match wordt de entiteit handmatig gevalideerd in het systeem. Indien de namen niet overeenkomen, wordt alles gereset en wordt het bedrijf op de hoogte gebracht om de gegevens opnieuw in te voeren.
    \item \textbf{Dataretentie:} Na enige tijd wordt de kopie van de identiteitskaart handmatig of na een ingestelde periode gewist om te voldoen aan de GDPR-richtlijnen.
\end{enumerate}

\begin{figure}[ht]
    \centering
    \begin{tikzpicture}[node distance=1.2cm and 0.6cm]

        %lanes
        \draw[thick] (-1, 2.5) -- (12, 2.5); 
        \draw[thick] (-1, 0.5) -- (12, 0.5); 
        \draw[thick] (-1, -4.5) -- (12, -4.5); 
        \node[rotate=90, font=\bfseries] at (-0.7, 1.5) {Klant};
        \node[rotate=90, font=\bfseries] at (-0.7, -2) {Scrada};

        %flow
        \node (start) [start] at (0, 1.5) {};
        \node (input) [task] [right=0.5cm of start] {Invoer data \& upload ID};
        
        \node (notif) [task] at (1.5, -1) {Ontvangst notificatie};
        \node (check) [task] [right=0.7cm of notif] {Manuele check KBO / Buitenland};
        \node (compare) [task] [right=0.7cm of check] {Match ID vs. Bestuurders};
        \node (decision) [gateway] [right=0.7cm of compare] {Match?};
        
        \node (approve) [task] [below=1cm of decision] {Validatie in systeem};
        \node (reset) [task] at ($(decision) + (2, 2.5)$) {Reset \& verwittig klant};
        
        \node (delete) [task] [left=1.5cm of approve] {Wissen ID (GDPR)};
        \node (end) [end] [left=1cm of delete] {};

        %connecties
        \draw [arrow] (start) -- (input);
        \draw [arrow] (input) -- (notif);
        \draw [arrow] (notif) -- (check);
        \draw [arrow] (check) -- (compare);
        \draw [arrow] (compare) -- (decision);
        \draw [arrow] (decision) -- node[anchor=east, font=\scriptsize] {Ja} (approve);
        \draw [arrow] (decision.east) -| node[anchor=south west, font=\scriptsize] {Nee} (reset);
        \draw [arrow] (reset) -- (input);
        \draw [arrow] (approve) -- (delete);
        \draw [arrow] (delete) -- (end);

    \end{tikzpicture}
    \caption[BPMN Huidig Proces]{BPMN-visualisatie van het huidige manuele KYC-proces bij Scrada (AS-IS).}
    \label{fig:bpmn-huidig}
\end{figure}

\section{Het Geautomatiseerde KYC-proces (eKYC)}
In vergelijking met het manuele KYC-proces streeft een geautomatiseerd proces naar een "seamless" onboarding, wat zorgt voor een hogere conversie. Volgens \textcite{Deloitte2024} is dit essentieel voor de klanttevredenheid
terwijl men ook aan de AML-wetgeving voldoet. 

Bij een geautomatiseerd proces (vaak eKYC genoemd) worden de menselijke handelingen vervangen door API-aanroepen en digitale identiteitsverificatie. 
De stappen verlopen doorgaans als volgt \autocite{ItsmeKYC2024, Deloitte2024}:

\begin{enumerate}
    \item \textbf{Digitale Identiteitsverificatie:} In plaats van een handmatige upload van een ID-kaart, gebruikt de klant een digitale ID zoals itsme®. 
    Hierdoor worden geverifieerde gegevens (naam, geboortedatum, etc.) direct en foutloos in het systeem ingeladen.
    \item \textbf{Geautomatiseerde Data-extractie en Matching:} Het systeem voert in real-time een API-call uit naar externe bronnen zoals de KBO. 
    De software vergelijkt automatisch of de persoon in de onboarding voorkomt in de lijst van statutaire bestuurders van de onderneming.
    Een voorbeeld van gestructureerde JSON-data die na een dergelijke controle wordt ontvangen, is te zien in Listing \ref{lst:api-response}.
    \item \textbf{Instant Risicobeoordeling:} De verzamelde data wordt direct getoetst aan sanctielijsten en PEP-lijsten (Politically Exposed Persons) via databases zoals LSEG World-Check. 
    Dit gebeurt zonder dat een medewerker handmatig hoeft te zoeken.
    \item \textbf{Automatische Statusupdate:} Indien alle controles positief zijn, wordt de status in het systeem onmiddellijk omgezet naar "approved". 
    Alleen bij onduidelijkheden of een verhoogd risicoprofiel wordt een medewerker gevraagd om de "exception" handmatig te beoordelen.
    \item \textbf{Veilige Gegevensopslag:} De data wordt direct versleuteld opgeslagen en er wordt automatisch een auditlog bijgehouden voor compliance-doeleinden, wat de kans op GDPR-inbreuken minimaliseert.
\end{enumerate}

\begin{listing}[ht]
\begin{minted}{json}
{
  "enterprise_number": "0123.456.789",
  "status": "Active",
  "sanction_check": "Cleared",
  "ubo_identified": true,
  "risk_score": "Low"
}
\end{minted}
\caption{Voorbeeld van een JSON API-response voor KYC-verificatie.}
\label{lst:api-response}
\end{listing}

\begin{table}[h!]
\centering
\caption{Vergelijking tussen manuele KYC en geautomatiseerde eKYC (gebaseerd op Deloitte en de Bolero case study)}
\label{tab:kyc-vergelijking}
\begin{tabular}{|l|p{5cm}|p{5cm}|}
\hline
\textbf{Kenmerk} & \textbf{Manueel proces (Scrada huidig)} & \textbf{Geautomatiseerd proces (eKYC / itsme®)} \\ \hline
\textbf{Klantervaring} & Hoog-frictie: handmatige uploads en wachttijden. & \textit{Seamless}: onmiddellijke verificatie via app \autocite{ItsmeBolero2024}. \\ \hline
\textbf{Data-input} & Foutgevoelig: handmatige overdracht van ID-gegevens. & Foutloos: directe overdracht van geverifieerde data \autocite{ItsmeKYC2024}. \\ \hline
\textbf{Doorlooptijd} & Dagen: afhankelijk van beschikbaarheid medewerker. & Seconden: real-time besluitvorming \autocite{Deloitte2024}. \\ \hline
\textbf{Bedrijfscheck} & Manuele opzoeking in KBO-webportaal door medewerker. & Automatische matching via KBO-API koppeling. \\ \hline
\textbf{Schaalbaarheid} & Moeilijk: werkdruk stijgt lineair met aantal klanten. & Hoog: systeem kan grote volumes simultaan verwerken. \\ \hline
\end{tabular}
\end{table}