\chapter{\IfLanguageName{dutch}{Stand van zaken}{State of the art}}%
\label{ch:stand-van-zaken}

% Tip: Begin elk hoofdstuk met een paragraaf inleiding die beschrijft hoe
% dit hoofdstuk past binnen het geheel van de bachelorproef. Geef in het
% bijzonder aan wat de link is met het vorige en volgende hoofdstuk.

% Pas na deze inleidende paragraaf komt de eerste sectiehoofding.

In het vorige hoofdstuk is de noodzaak voor een efficiënter en veiliger KYC-proces bij Scrada aangetoond. 
Om een correct werkende automatisering te kunnen ontwerpen, is een grondig begrip van zowel de juridische voorwaarden als de beschikbare technologische bouwstenen essentieel. 
Dit hoofdstuk onderzoekt het Peppol-ecosysteem, de strenge vereisten van de GDPR en de moderne technologieën die identiteitsverificatie kunnen digitaliseren en efficiënter maken. 
Dit vormt de theoretische fundering voor de Proof-of-Concept in de latere hoofdstukken.

\label{sec:Juridisch Kader: Het spanningsveld tussen KYC en GDPR}

Bij KYC-processen worden per definitie persoonsgegevens verwerkt. 
Het is een kritieke procedure die financiële instellingen en bedrijven helpt om de identiteit van hun klanten te verifiëren, 
wat essentieel is voor het voorkomen en identificeren van fraude, witwassen van geld en andere illegale activiteiten.
Hierdoor is het KYC-proces internationaal gestandaardiseerd.

Echter, de identificatie en privacyrechten van de klant moeten in evenwicht worden gebracht, wat een uitdaging vormt voor bedrijven zoals Scrada die aan de GDPR-regelgeving moeten voldoen.
In deze sectie worden de belangrijkste juridische principes van de GDPR besproken die relevant zijn voor het KYC-proces, 
evenals de uitdagingen en mogelijke oplossingen voor het implementeren van een geautomatiseerd KYC-systeem dat voldoet aan deze regelgeving.

\subsection{Principes van de GDPR}
Om een geautomatiseerd KYC-proces te creëren dat juridisch conform is aan de GDPR, 
moet er specifiek rekening worden gehouden met de kernprincipes van de Europese privacywetgeving \autocite{EuropeanUnion2016}. 
Voor dit onderzoek staan de volgende artikelen centraal:
\autocite{EuropeanUnion2016}:
\begin{itemize}
    \item \textbf{Dataminimalisatie (Art. 5.1c):} TODO
    \item \textbf{Doelbinding (Art. 5.1b):} TODO
    \item \textbf{Opslagbeperking (Art. 5.1e):} TODO
    \item \textbf{Geautomatiseerde besluitvorming (Art. 22):} TODO
    \item \textbf{Privacy by Design (Art. 25):} TODO
\end{itemize}

\label{sec:Het gestandaardiseerde KYC-proces}
Volgens de \textcite{LSEG2024} bestaat een effectief proces uit drie essentiële stappen: 
het identificeren van de klant (CIP), het uitvoeren van een gepast cliëntenonderzoek (CDD) en de doorlopende monitoring van de zakelijke relatie.
Dit gebeurt bij de onboarding van nieuwe klanten (de eerste fase van de customer journey)
maar ook bij het onderhouden van de huidige klantenrelaties (de latere fases van de customer journey).

\begin{figure}
  \centering
  \includegraphics[width=0.8\textwidth]{grail.jpg}
  \caption[Voorbeeld figuur.]{\label{fig:grail}Voorbeeld van invoegen van een figuur. Zorg altijd voor een uitgebreid bijschrift dat de figuur volledig beschrijft zonder in de tekst te moeten gaan zoeken. Vergeet ook je bronvermelding niet!}
\end{figure}

\begin{listing}
  \begin{minted}{python}
    import pandas as pd
    import seaborn as sns

    penguins = sns.load_dataset('penguins')
    sns.relplot(data=penguins, x="flipper_length_mm", y="bill_length_mm", hue="species")
  \end{minted}
  \caption[Voorbeeld codefragment]{Voorbeeld van het invoegen van een codefragment.}
\end{listing}

\lipsum[7-20]

\begin{table}
  \centering
  \begin{tabular}{lcr}
    \toprule
    \textbf{Kolom 1} & \textbf{Kolom 2} & \textbf{Kolom 3} \\
    $\alpha$         & $\beta$          & $\gamma$         \\
    \midrule
    A                & 10.230           & a                \\
    B                & 45.678           & b                \\
    C                & 99.987           & c                \\
    \bottomrule
  \end{tabular}
  \caption[Voorbeeld tabel]{\label{tab:example}Voorbeeld van een tabel.}
\end{table}

