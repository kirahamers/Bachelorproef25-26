%%=============================================================================
%% Samenvatting
%%=============================================================================

% TODO: De "abstract" of samenvatting is een kernachtige (~ 1 blz. voor een
% thesis) synthese van het document.
%
% Een goede abstract biedt een kernachtig antwoord op volgende vragen:
%
% 1. Waarover gaat de bachelorproef?
% 2. Waarom heb je er over geschreven?
% 3. Hoe heb je het onderzoek uitgevoerd?
% 4. Wat waren de resultaten? Wat blijkt uit je onderzoek?
% 5. Wat betekenen je resultaten? Wat is de relevantie voor het werkveld?
%
% Daarom bestaat een abstract uit volgende componenten:
%
% - inleiding + kaderen thema
% - probleemstelling
% - (centrale) onderzoeksvraag
% - onderzoeksdoelstelling
% - methodologie
% - resultaten (beperk tot de belangrijkste, relevant voor de onderzoeksvraag)
% - conclusies, aanbevelingen, beperkingen
%
% LET OP! Een samenvatting is GEEN voorwoord!

%%---------- Nederlandse samenvatting -----------------------------------------
%
% TODO: Als je je bachelorproef in het Engels schrijft, moet je eerst een
% Nederlandse samenvatting invoegen. Haal daarvoor onderstaande code uit
% commentaar.
% Wie zijn bachelorproef in het Nederlands schrijft, kan dit negeren, de inhoud
% wordt niet in het document ingevoegd.

\IfLanguageName{english}{%
\selectlanguage{dutch}
\chapter*{Samenvatting}
\lipsum[1-4]
\selectlanguage{english}
}{}

%%---------- Samenvatting -----------------------------------------------------
% De samenvatting in de hoofdtaal van het document

\chapter*{\IfLanguageName{dutch}{Samenvatting}{Abstract}}
De digitalisering van financiële processen brengt problemen met zich mee voor dienstverleners zoals Scrada, een Certified Peppol Access Point, vooral vanwege de strenge GDPR-regels. 
Het huidige KYC-proces (Know Your Customer) van Scrada is niet efficiënt, want het handmatig controleren van identiteitskaarten met de KBO-databank zorgen voor excessieve dataprocessing en een hoog risico op fouten. 
% TODO extra informatie over hoe het huidige proces eruit ziet, en waarom het niet voldoet aan de eisen van GDPR, fraudeveiligheid en kostenefficiëntie.
Daarom is de centrale onderzoeksvraag van deze bachelorproef: Hoe kan het KYC-proces bij de identiteitsverificatie van bedrijven op het Peppol-platform van Scrada geautomatiseerd worden op een GDPR-conforme, fraudeveilige en kostenefficiënte manier? 
Het concrete doel is om een werkend Proof-of-Concept (PoC) te ontwikkelen voor deze automatisering en deze uiteindelijk te integreren bij de applicatie van Scrada. 
Dit aan de hand van een methodologie die literatuurstudie combineert met een BPMN-analyse van het huidige en toekomstige proces 
samen met de implementatie van technologieën zoals Blockchain (One-Time Verification) en AI. 
Het doel hiermee is om de PoC de KYC-doorlooptijd met minimum 50 procent zal inkorten, de operationele kosten te verlagen door het vermijden van dure externe licenties, 
en tegelijkertijd de GDPR-conformiteit en fraudeveiligheid bewijsbaar met testen te verhogen. De meerwaarde voor Scrada is dat dit onderzoek de KYC-procedure verandert van een knelpunt in een 
strategisch en efficiënt voordeel in de markt. Dit werd bereikt aan de hand van %TODO methodologie
De resultaten die uit dit onderzoek naar voren kwamen, %TODO resultaten
Als conclusie kan worden gesteld dat de %TODO conclusie

TODO INFO PROCES, METHODOLOGIE, RESULTATEN, CONCLUSIE
